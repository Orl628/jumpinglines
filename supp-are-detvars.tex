\begin{proposition} \label{supp-are-det-varieties}
The sets
$\Supp(R^1 \pushf{\phi}\pullb{p} V_{k}(-b^{(k)}-1))$ and
$\Supp(R^1 \pushf{\phi}(\pullb{p} V_{k}(-b^{(k)})\dual))$ 
are determinantal varieties in the sense of \cite[Ch.~II, §4]{arbarello-geometry-algebraic-curves}
\end{proposition}
\begin{proof}
To simplify notation, set
$r_1 \coloneqq \dim H^0(\PP^n, \mc{O}(k)),
r_2 \coloneqq \dim H^0(\PP^n, \mc{O}(k-d))$ and $b\coloneqq b^{(k)}$,
and rewrite the exact sequence from \Cref{verlinde-exact-sequence} as
\begin{align} \label{verlinde-simplified-exact-sequence}
\ses{\mc{O}(-1)^{r_2}}{\mc{O}^{r_1}}{V_k}. %\tag{$\star$}
\end{align}
Twisting the sequence \cref{verlinde-simplified-exact-sequence} with $\mc{O}(-b-1)$ and pulling back to $P$ gives an exact sequence
\[
0
\to  {\pullb{p}\mc{O}(-b-2)^{r_2}}
\to  {\pullb{p}\mc{O}(-b-1)^{r_1}}
\to  {\pullb{p}V_k(-b-1)}
\to  0.
\]
For all $t\in \GGr(1,\schemeofsurfaces)$ we have
$h^2(P_t, \mc{O}(-b-2)^{r_2}) = 0$,
hence
$R^2\pushf{\phi}\pullb{p}\mc{O}(-b-2)^{r_2} = 0$
and applying $\pushf{\phi}$ to the above sequence gives an exact sequence
\[
R^1\pushf{\phi}\pullb{p}\mc{O}(-b-2)^{r_2}
\xto{\alpha}
R^1\pushf{\phi}\pullb{p}\mc{O}(-b-1)^{r_1} 
\to
R^1\pushf{\phi}\pullb{p} V_k(-b-1)
\to 0.
\]
Note that since the numbers
$
h^{1}_{2}\coloneqq h^1(P_t, \mc{O}(-b-2)^{r_2})
\text{ and }
h_{1}^{1}\coloneqq h^1(P_t, \mc{O}(-b-1)^{r_1})
$
do not depend on the point $t$, Grauert's Theorem applies, and the first two terms of the above sequence are locally free and coherent of rank $h_1^2$ and $h_1^1$, respectively. Since taking the fiber is right-exact, we see that for all $t$ we have
$(R^1\pushf{\phi}\pullb{p} V_k(-b-1))_t \neq 0$ if and only if $\coker(\alpha_t) \neq 0$. Concluding, we have
\[
\Supp(R^1\pushf{\phi}(\pullb{p} V_k(-b-1)))
= \{t : \rank (\alpha_t)\leq h^{1}_1 - 1\}.
\]
As a final remark, note that $h^1_1 = b r_1 = b \binom{k+n}{n}.$

The proof for the second assertion is similar. We start with the sequence \cref{verlinde-simplified-exact-sequence}, twist with $\mc{O}(-b)$, take duals, pull back to $P$, and apply $\pushf{\phi}$. Since for all $t\in \GGr(1,\schemeofsurfaces)$ we have $h^1(P_t, \mc{O}(b)^{r_1})=0$, we obtain an exact sequence
\[
	{\pushf{\phi}\pullb{p}\mc{O}(b)^{r_1} }
\xto{\beta}	{\pushf{\phi}\pullb{p}\mc{O}(b+1)^{r_2}}
\to	{R^1 \pushf{\phi}(\pullb{p} V_{k}(-b)\dual)}
\to 0.
\]
Since the numbers
$
h^0_1 \coloneqq h^0(P_t,\mc{O}(b)^{r_1}) \text{ and }
h^0_2 \coloneqq h^0(P_t, \mc{O}(b+1)^{r_2})
$
do not depend on the point $t$, again by Grauert's Theorem the first two terms of the sequence are locally free of rank $h^0_1$ and $h^0_2$, respectively. As before, we obtain the characterization
\[
	\Supp(R^1 \pushf{\phi}(\pullb{p} V_{k}(-b)\dual))
	= \{t : \rank (\beta_t)\leq h^{0}_2 - 1\}.
\]
Here, we have $h_2^0 = (b+2)r_2 = (b+2)\binom{k+n-d}{n}.$
% Gilt Serre-dualität auch für Familien?
\end{proof}
