%!TEX root = ./master.tex

%\begin{definition} A \emph{line} is an embedding $l\colon \mathbb{P}^1 \to X$ into a projective scheme $X$ such that $l^*\mathcal{O}(1) = \mathcal{O}(1)$.
%\end{definition}
\begin{definition}
Let $\pi\from \mf X \to \abs{\mc{O}_{\PP^n}(d)}$ be the universal family of hypersurfaces of degree $d$ in $\PP^n$. Let $\mc{L}$ be the restriction to $\mathfrak X$ of the bundle $\mc{O}(1)\boxtimes \mc{O}$ under the inclusion $\mf X \subseteq \PP^n \times \schemeofsurfaces$.
\end{definition}

For $k\geq 1$, the sheaf $\pushf{\pi}\mc{L}^{\otimes k}$ is locally free of rank $r^{(k)}\coloneqq \binom{k+n}{n} - \binom{k+n-d}{n}$, as can be seen by considering the structure sequence of an arbitrary hypersurface of degree $d$ in $\PP^n$.  

% which can be seen as follows. Let the index $I$ range over the tuples of the 
% form $(i_0,i_1,i_2,i_3)$ with $i_j \geq 0$ and $\sum i_j = 4$, and let $x_I$ 
% denote the $I$-th projective coordinate of $\schemeofsurfaces$. For $j=0,\dotsc
% ,3$, let $x_j$ denote the $j$-th coordinate of $\PP^n$. Then the family $\mf{X}
% $ is cut out by the section $\sum_{I} x_{I} x^{I}$ of the line bundle $\mc{O}(4
% ) \boxtimes \mc{O}(1)$ on $\PP^n \times \schemeofsurfaces$.

%\begin{proposition} \label{verlinde-base-change}
	Let $k\geq 1$. The following statements hold:

	\begin{enumerate}
	\huyitem If $q\in \schemeofsurfaces$ then
	$h^0 ( \mf{X}_q, \mc{L}^{\otimes k}|_q) = \binom{k+n}{n} - \binom{k+n-d}{n}$.
	In particular, this number is independent of the point $q$.

	\huyitem The sheaf
	$\pushf{\pi}\mc{L}^{\otimes k}$
	is locally free of rank
	$\binom{k+n}{n} - \binom{k+n-d}{n}$.

	\huyitem For all cartesian diagrams of the form 
	\[
	\cartesiansquare{\mf{X}_{Z}}{}{\mf{X}}{\pi_Z}{\pi}{Z}{\rho}{\schemeofsurfaces}
	\]
	we have
	$\pullb{\rho}\pushf{\pi}\mc{L}^{\otimes k}
	\simeq
	\pushf{(\pi_Z)}\mc{L}^{\otimes k}_Z$.
	\end{enumerate}
\end{proposition}

\begin{proof}
	The proof for the first statement is found in
	\cite[Proposition 4.1]{hemminghaus-verlinde-bundles}
	and reproduced below. The others follow from Grauert's Theorem
	\cite[{}28.1.5]{vakil-algebraic-geometry}.

	Let $X\coloneqq \mathfrak X_q$ be the hypersurface of degree $d$ corresponding to the point $q$. We have $\mc{L}^{\otimes k}|_q = \mc{O}_{\PP^n}(k)|_{X}$. Twisting the structure sequence
	\[\ses{\mc{O}(-d)}{\mc{O}}{\mc{O}_{X}}\]
	on $\PP^n$
	with $\mc{O}(k)$ yields the short exact sequence
	\[\ses{\mc{O}(k-d)}{\mc{O}(k)}{\mc{L}^{\otimes k}|_q}.\]
	The statement follows by taking global sections and using $H^1(\mc{O}(k-d),\PP^n)=0$ for $n>1$.
	% Todo: replace with own short proof
	% of the sections of the sheaf $\mc{L}$ at the fiber of a point $q\in \schemeofsurfaces$ does not depend on $q$. Indeed, the fiber $\mc{X}_q$ is a hypersurface of degree $4$ embedded in the projective space $\PP^{3}_{\kappa(q)}$. Its structure sequence on $\PP^{3}_{\kappa(q)}$ is
	% \[\ses{\mc{O}(-d)}{\mc{O}}{\mc{O}_{X_q}}.\]
	% Twisting with $\mc{O}(k)$ yields
	% \[\ses{\mc{O}(k-d)}{}{}\]
\end{proof}


\begin{definition}Let $k\geq 1$. The $k$-th \emph{Verlinde bundle} of the family $\pi$ is the vector bundle
$V_k \coloneqq \pushf{\pi}\mc{L}^{\otimes k}$.
\end{definition}

%\begin{proposition} \label{verlinde-exact-sequence}
	There exists a short exact sequence of vector bundles on $\schemeofsurfaces$ 
	\begin{equation} \label{master-verlinde-sequence}
	0\to  \mc{O}(-1) \otimes H^0(\PP^n, \mc{O}(k-d)) 
	\xto{M}  \mc{O} \otimes {H^0(\PP^n, \mc{O}(k))   }
	\to{V_k}
	\to 0.
	\end{equation}
	The map $M$ is given by multiplication by
	$\sum_I \alpha_I \otimes x^I \in
	H^0(\schemeofsurfaces,\mc{O}(1)) \otimes H^0(\PP^n,\mc{O}(d)).$
	\end{proposition}

\begin{proof}
	The proof, found in \cite[Proposition 4.2]{hemminghaus-verlinde-bundles}, is reproduced below.

	The structure sequence of $\mathfrak X$ on $\PP^n \times \schemeofsurfaces$ is
	\[
		\ses{\mc{O}_{\PP^n}(-d)\boxtimes \mc{O}_{\schemeofsurfaces}(-1)}
		{\mc{O}_{\PP^n}\boxtimes \mc{O}_{\schemeofsurfaces}}
		{\mc{O}_{\mathfrak X}},
	\]
	the first map given by multiplication with $\sum_I \alpha_I \otimes x^I$.
	Twisting with $\mc{L}^{\otimes k}=\mc{O}(k)\boxtimes \mc{O}$, we get the exact sequence
	\[
		\ses{\mc{O}(k-d)\boxtimes \mc{O}(-1)}
		{\mc{O}(k)\boxtimes \mc{O}}
		{\mc{L}^{\otimes k}|_X}.
	\]
	Applying the pushforward 
	$\pi_*$
	we get the sequence \cref{master-verlinde-sequence}, 
	which is exact as
	\[
		R^1 \pi_* (\mc{O}(k-d)\boxtimes \mc{O}(-1)) = 0.
	\]
	The description of the map $M$ follows by the definition of the pushforward.
\end{proof}

There exists a short exact sequence of vector bundles on $\schemeofsurfaces$ 
	\begin{equation} \label{master-verlinde-sequence}
	0\to  \mc{O}(-1) \otimes H^0(\PP^n, \mc{O}(k-d)) 
	\xto{M}  \mc{O} \otimes {H^0(\PP^n, \mc{O}(k))   }
	\to{V_k}
	\to 0,
	\end{equation}
as can be seen by taking the pushforward of a twist of the structure sequence of $\mathfrak X$ on $\PP^n \times \schemeofsurfaces$.
	The map $M$ is given by multiplication by the section
	$$\sum_I \alpha_I \otimes x^I \in
	H^0(\schemeofsurfaces,\mc{O}(1)) \otimes H^0(\PP^n,\mc{O}(d)).$$

\begin{definition} Let $T\subseteq \schemeofsurfaces$ be a line. On $T=\PP^1$, we define the vector bundle $V_{k,T}\coloneqq V_k|_T$. The \emph{splitting type} of $V_{k,T}$ is the unique non-increasing tuple $(b_1,\dotsc, b_{r^{(k)}})$ such that $V_{k,T} \simeq \bigoplus_i \mc O(b_i)$.
\end{definition}

The sequence \cref{master-verlinde-sequence} puts constraints on the $b_i$: they are all non-negative and they sum up to $d^{(k)}\coloneqq\deg (V_k) = \binom{k+n-d}{n}$. The set of such tuples $(b_i)$ can be ordered by defining the expression $(b'_i) \geq (b_i)$ to mean
	\[
		\sum_{i=1}^s b'_i \geq \sum_{i=1}^s b_i \text{ for all $s=1,\dotsc, r$}.
	\]
With this definition, smaller types are more general: the vector bundle $\mc O (b_i)$ on $\PP^1$ specializes to $\mc O (b'_i)$ in the sense of \cite{schatz-degeneration-specialization} if and only if $(b'_i) \geq (b_i),$ see for example \cite{ramamathan-deformations}. 

If $d^{(k)} \leq r^{(k)}$, then the most generic possible type has thus the form $(1,\dotsc,1,0,\dotsc,0)$. We call this the \emph{generic splitting type}. A computation shows that $d^{(k)} \leq r^{(k)}$ if $k\leq 2d$.

\Cref{number-zeroes} shows that we can recover partial information about the type of $V_{k,T}$ by looking at two polynomials spanning the line $T$. More precisely, it lets us count the number of non-zero entries of the type. While this information does not suffice to discern between different types, it does characterize the generic type and the second most generic type, $(2,1,\dotsc,1,0,\dotsc,0)$.

In \Cref{thm:cohomology-class}, we calculate the cohomology class of the closed subvariety
\[
	Z\coloneqq \{T\in \GGr(1,\schemeofsurfaces)\mid V_{d+1,T} \text{ has non-generic type}\}
\]
of the Grassmannian of lines in $\schemeofsurfaces$. Aside from the restriction $k=d+1$, we also have to assume $n\leq 3$. The computation is carried out by the method of undetermined coefficients, leading into various calculations in the Chow ring of the Grassmannian. The assumption $n\leq 3$ is needed for a certain dimension estimation.

\subsection*{Aknowledgement}
This work is a condensed version of my Master's thesis, supervised by Daniel Huybrechts. I would like to take the opportunity to thank him for his mentorship during the writing of the thesis, as well as for his help during the preparation of this article.

% \begin{remark}
% 	For $k<d$, the sequence \cref{master-verlinde-sequence} shows that $V_k$ is trivial. For $k=d,$ the sequence \cref{master-verlinde-sequence} is the Euler sequence and $V_k=\mc{O}(-1)$.
% \end{remark}


% \item A \emph{type candidate} for $V_k$ is a tuple \footnotemark{}
% $(d_1,\dotsc,d_{r(k)})$
% of non-negative integers
% with $r(k)=\binom{k+n}{n}-\binom{k+n-d}{n}$ and
% $\sum b_i = \binom{k+n-d}{n}$.

% \footnotetext{i.e.\ an equivalence class of tuples up to reordering of the entries}

% \item Of the type candidates for $V_k$, there exists a unique one of the form
% $(d_k,\dotsc,d_k,d_{k+1},\dotsc,d_{k+1})$, namely with $d_k=$
% The \emph{general type candidate} for $V_k$ is the unique\footnotemark{} type candidate for $V_k$ of the form
% $(d,\dotsc,d,d+1,\dotsc, d+1)$.
% %maybe mention relationship with specialization on ext^1.
% \footnotetext{For fixed $r,s>0$, the equation $dr + b = s$ has only one non-negative solution $(d,b)$ with $b<r$.}


% The points of $\Gr(2,35)$ correspond to the pencils of quartics $T \subseteq \schemeofsurfaces$ in the following way. Let $P$ the universal $\PP^1$-bundle over $\Gr(2,35)$. It comes equipped with a projection map $P \to \schemeofsurfaces$ such that for all pencils of quartics $T$ there exists a unique point $t\in \Gr(2,35)$
% such that the image of the fiber $P_t$ in $\schemeofsurfaces$ is $T$.
% \[
% \begin{tikzcd} [ampersand replacement = \&]
% P_t \arrow{r} \arrow{d} \MySymb{\times}{dr} \& P \arrow{r}{p} \arrow{d}{\phi}\&
% \schemeofsurfaces \\
% \Spec(\kappa(t))\arrow{r} \& \Gr(2,35) \& \ 
% \end{tikzcd}
% \]
% For $t\in\Gr(2,35)$ corresponding to the pencil $T$, we write $V_{k,t}\coloneqq V_{k,T}$.



% \begin{proposition}	
% The set $Z_{\text{gen}}$, and its complement, are not empty.
% \end{proposition}

% \begin{proposition}
% The set $Z_{\text{gen}}$ is Zariski-open. Its complement is a determinantal variety of codimension at least ()().
% \end{proposition}

% \begin{proof}
% After dualizing and pulling back the exact sequence from
% \Cref{verlinde-exact-sequence},



% The map $\phi$ is flat and proper, the scheme $\Gr(2,35)$ is reduced and locally Noetherian.
% \end{proof}

%%TODO	[-> Harder-Narasimhan-Polynome]
% \begin{proposition}
% 	Let $(b_i)$ be a type candidate for $V_k$. The set $\widehat Z_{(b_i)} \coloneqq \bigcup_{(b'_i)\geq(b_i)} Z_{(b'_i)}$ is the intersection of (at most) $r^{(k)}$ determinantal varieties. In particular, the set $Z_{(b_i)}$ is locally closed.
% \end{proposition}

% \begin{proof}
% Let $t\in \GGr(1,\schemeofsurfaces)$ and $V_{k,t} = \bigoplus_{i=1}^{r^{(k)}} \mc{O}(b'_i)$. We have
% \[
% \bigwedge^s V_{k,t} = \bigoplus_{I} \mc{O}(d_I'),
% \]
% where $I$ runs over the subsets of $\{1,\dotsc, r^{(k)}\}$ of size $s$ and $d'_I\coloneqq \sum_{i\in I} b'_i$.
% For every type candidate $(b'_i)$, the sum $\sum_{i=1}^s b'_i$ is the largest sum of $s$ entries of $(b'_i)$. Since $b'_i \geq 0$, the condition $\sum_{i=1}^s b'_i \geq \sum_{i=1}^s b_i$ is equivalent to the condition $h^0((\textstyle{\bigwedge}^{s} V_{t,k})(-\textstyle{\sum}^s b_i)) > 0$.
% \[
% 	\widehat Z_{(b_i)} = \bigcap_{s=1}^{r^{(k)}} \{t : h^0((\textstyle{\bigwedge}^{s} V_{t,k})(-\textstyle{\sum}^s b_i)) > 0\}.
% \]
% With Serre duality and the Cohomology and Base Change theorem we write the sets of the intersection as 
% \[
% 	\Supp(R^1 \pushf \phi (\pullb p (\textstyle{\bigwedge}^{s} V_k\dual )(\textstyle{\sum}^s b_i - 2))),
% \]
% which is a determinantal variety by an argument similar to the second part of the proof of \Cref{supp-are-det-varieties}: start with the sequence \cref{verlinde-simplified-exact-sequence},  One just has to note that $h^1(\PP^1,\mc{O}(\sum^s b_i - 2)) = 0$ for all $s$, since at least one $b_i$ is nonzero. This gives the exact sequence
% % \[
% % {\pushf{\phi}\pullb{p}\mc{O}(b)^{r_1} }
% % \xto{\beta}	{\pushf{\phi}\pullb{p}\mc{O}(b+1)^{r_2}}
% % \to	{R^1 \pushf{\phi}(\pullb{p} V_{k}(-b)\dual)}
% % \to 0.
% % \]
% \end{proof}

