%!TEX root = ./master.tex
\section{Introduction}

% \subsection{Notation and conventions}
% Throughout, $k$ will denote an algebraically closed field, but we omit it from most notation. The letter $k$ will also denote a natural number.

% For natural numbers $d$ and $n$, we write $I_d$ for a tuple of non-negative integers of the form $(i_0,\dotsc,i_n)$ with
% $\sum i_j = d$. Thus for example a tuple ranging over the $I_d$ will have $\binom{n+d}{n}$ entries.

% We fix names for the homogeneous coordinates of various projective spaces: for the coordinates of $\PP^1$ we write $s$ and $t$, for $\PP^n$ we write $x_i$, and for the coordinates of $\schemeofsurfaces$ we take $\alpha_{I_d}$, where we think of $\alpha_{I_d}$ as corresponding to $x^{I_d}\coloneqq \prod_{i} x_i^{(I_d)_i}$.

% For a fiber product $X \xleftarrow{p} X\times Y\xrightarrow{q} Y$ and sheaves $\mc{F}$ and $\mc{G}$ on $X$ resp.\ $Y$, we write $\mc{F}\boxtimes \mc{G}\coloneqq \pullb{p} \mc{F} \otimes \pullb{q} \mc{G}$.

\subsection{Aknowledgement}
This work is a condensed version of my Master's thesis, supervised by Daniel Huybrechts. I would like to take the opportunity to thank him for his mentorship during the writing of the thesis, as well as for his help during the preparation of this article.